\documentclass[a4paper,10pt,twoside,openany]{book}

\usepackage[lang=hebrew]{maths}
\usepackage{hebrewdoc}
\usepackage{stylish}
\usepackage{lipsum}
\let\bs\blacksquare

\setlength{\parindent}{0pt}

%%%%%%%%%%%%
% Styling %
%%%%%%%%%%%%

\usepackage{enumitem}

%%%%%%%%%%%%%
% Counters  %
%%%%%%%%%%%%%

\setcounter{section}{1}     
            
%BIBLIOGRAPHY
\usepackage[
backend=biber,
style=alphabetic,
]{biblatex}
\addbibresource{bibliography.bib} %Imports bibliography file

\title{
\includegraphics[width=6in]{images/front.png}\\
\vspace{30pt}
\Huge
מבוא לתורת המספרים (104157)
\\
אביב 2024
\\
רשימות תרגולים
\vspace{30pt}
\\
\huge
אלן סורני
\vspace{30pt}
\\
\Large
הרשימות עודכנו לאחרונה בתאריך ה־%
\today
}
\date{}

\begin{document}
\frontmatter
\maketitle
\tableofcontents

\mainmatter

\section*{סימונים}

\begin{itemize}
\item[-]
$\mbb{N} = \set{0, 1, 2, \ldots}$
אוסף המספרים הטבעיים.
\item[-]
$\mbb{N}_+ = \set{1, 2, 3, \ldots}$
אוסף המספרים הטבעיים החיוביים (כלומר, לא כולל אפס).
\item[-]
$\brs{n} = \set{1, \ldots, n}$.
\item[-]
$\floor{x}$
המספר הכי גדול שקטן או שווה ל־%
$x \in \mbb{R}$.
\item[-]
$\ceil{x}$
המספר הכי קטן שגדול או שווה ל־%
$x$.
\item[-]
\begin{align*}
\gcd\prs{a_1, \ldots, a_n} \\
\lcm\prs{a_1, \ldots, a_n}
\end{align*}
בהתאמה, המחלק המשותף הגדול ביותר של המספרים
$a_1, \ldots, a_n$,
והכפולה המשותפת המינימלית שלהם.
\end{itemize}

\chapter{תרגול 3 - שימושים בפריקות יחידה}

\section{תזכורת}

\begin{definition}
יהי
$n \in \mbb{N}_+$.
נגדיר
\begin{enumerate}
\item $\nu\prs{n} \coloneqq \sum_{d \mid n} 1$. זה מספר המחלקים של $n$.
\item $\sigma\prs{n} \coloneqq \sum_{d \mid n} d$. זה סכום המחלקים של $n$.
\item \[\text{.}\phi\prs{n} \coloneqq \sum_{\substack{\gcd\prs{d,n} = 1 \\ 1 < d < n}} 1\] זה מספר המספרים הטבעיים שקטנים מ־%
$n$
וזרים לו.
זאת נקראת
\textbf{פונקציית אוילר (\textenglish{Euer totient function})}.
\item $\pi\prs{n}$ מספר האיברים הראשוניים הקטנים או שווים ל־%
$n$.
זאת נקראת
\textbf{פונקציית המספרים הראשוניים (
\textenglish{prime-counting~function})}.
\item \[\mu\prs{n} = \begin{cases}
\prs{-1}^\ell & \text{\textenglish{$\forall p$ prime}}: p^2 \nmid n \\
0 & \text{otherwise}
\end{cases}\]
כאשר
$\ell$
מספר הראשוניים שמחלקים את
$n$.
זאת נקראת
\textbf{פונקציית מביוס (\textenglish{Möbius function})}.
\end{enumerate}
\end{definition}

\section{תרגילים}

\begin{exercisechap}[פרק 2, תרגיל 7]
הסיקו מתרגיל 6 כי
\[\ord_p\prs{n!} \leq \frac{n}{p-1}\]
וכי
\[\text{.} \sqrt[n]{n!} \leq \prod_{p \mid n!} p^{1 / \prs{p-1}}\]
\end{exercisechap}

\begin{solution}
לפי תרגיל 6 מהתרגול הקודם,
\begin{align*}
\text{.} \ord_p\prs{n!} &= \sum_{k = 1}^{\infty} \floor{\frac{n}{p^k}}
\end{align*}
נקבל כי
\begin{align*}
\ord_p\prs{n!} &\leq \sum_{k=1}^{\infty} \frac{n}{p^k}
\\&= n \sum_{k=1}^{\infty} \prs{\frac{1}{p}}^k
\end{align*}
וכיוון ש־%
$p \in \mbb{N}_+$
ראשוני מתקיים
$\abs{\frac{1}{p}} < 1$.
מסכום סדרה הנדסית נקבל כי
\begin{align*}
\sum_{k=1}^{\infty} \prs{\frac{1}{p}}^k
\\&= \frac{1 - \prs{1 - \frac{1}{p}}}{1 - \frac{1}{p}} - 1
\\&= \frac{\frac{1}{p}}{1 - \frac{1}{p}}
\\&= \frac{1}{p - 1}
\end{align*}
ולכן
\[\text{.}\ord_p\prs{n!} \leq \frac{n}{p - 1}\]

אז מתקיים גם
\begin{align*}
n! &= \prod_{\substack{p \mid n \\ \text{\textenglish{$p$ prime}}}} p^{\ord_p\prs{n!}}
\\&\leq
\prod_{\substack{p \mid n \\ \text{\textenglish{$p$ prime}}}} p^{\frac{n}{p-1}}
\\&\leq
\prod_{p \mid n} p^{\frac{n}{p-1}}
\\&=
\prs{\prod_{p \mid n} p^{\frac{1}{p-1}}}^n
\end{align*}
ולכן
\[\text{,} \sqrt[n]{n!} \leq \prod_{p \mid n} p^{\frac{1}{p-1}}\]
כנדרש.
\end{solution}

\begin{exercisechap}[פרק 2, תרגיל 8]
השתמשו בתוצאת התרגיל הקודם כדי להראות שיש אינסוף ראשוניים.

\emph{רמז:}
הראו קודם שמתקיים
$\prs{n!}^2 \geq n^n$
לכל
$n \in \mbb{N}_+$.
\end{exercisechap}

\begin{solution}
ראשית, נראה כי
$\prs{n!}^2 \geq n^n$
לכל
$n \in \mbb{N}_+$.

מתקיים
\begin{align*}
n! &= \prod_{k=0}^{n-1} \prs{k + 1} \\
n! &= \prod_{k=0}^{n-1} \prs{n - k}
\end{align*}
ולכן
\begin{align*}
\text{.} \prs{n!}^2 = \prod_{k=0}^{n-1} \prs{k+1}\prs{n-k}
\end{align*}

נראה כי הגורמים הנסכמים גדולים או שווים ל־%
$n$.
כאשר
$k = 0$
מתקיים
$\prs{k+1}\prs{n-k} = n$.
כאשר
$0 < k \leq \frac{n}{2}$
מתקיים
$n - k \geq \frac{n}{2}$
ואז
\[\text{.}\prs{k+1} \prs{n-k} \geq \frac{\prs{k+1} n}{2} \geq \frac{2n}{2} = n\]
כאשר
$\frac{n}{2} < k < n-1$
נקבל כי
$n - k \geq 2$
ולכן
\[\text{.} \prs{k+1}\prs{n-k} > 2 \cdot \frac{n}{2} = n\]
כאשר
$k = n-1$
נקבל
\[\text{.} \prs{k+1}\prs{n-k} = \prs{n-1+1} \cdot \prs{n-n+1} = n \cdot 1 = n\]
לכן
$\prs{n!}^2 \geq \prod_{k=0}^{n-1} n = n^n$,
כנדרש.

כעת, מהוכחת הסעיף הקודם ניתן לראות כי
\begin{align*}
\sqrt[n]{n!} \leq \prod_{\substack{p \mid n! \\ \text{\textenglish{$p$ prime}}}} p^{\frac{1}{p-1}}
\end{align*}
ואם נראה שאגף שמאל שואף לאינסוף נקבל שגם אגף ימין שואף לאינסוף, ובפרט שיש אינסוף ראשוניים.

אכן, מכך שמתקיים
$\prs{n!}^2 \geq n^n$
נובע כי
$\sqrt[n]{n!} \geq \sqrt{n}$
ולכן
$\lim_{n\to\infty}\sqrt[n]{n!} = \infty$.
\end{solution}

\begin{exercisechap}[פרק 2, תרגיל 15]
הראו כי
\begin{enumerate}[label = (\alph*)]
\item לכל
$n \in \mbb{N}_+$
מתקיים
\[\text{.} \sum_{d \mid n} \mu\prs{n/d} \nu\prs{d} = 1\]
\item לכל
$n \in \mbb{N}_+$
מתקיים
\[\text{.} \sum_{d \mid n} \mu\prs{n/d} \sigma\prs{d} = n\]
\end{enumerate}
\end{exercisechap}

\begin{solution}
ראשית נזכיר כי
\begin{align*}
\prs{f * g}\prs{n} \coloneqq \sum_{d \mid n} f\prs{d} g\prs{\frac{n}{d}}
\end{align*}
לכל
$f,g \colon \mbb{N}_+ \to \mbb{C}$,
וכי ראינו שלכל
$f$
כנ"ל מתקיים
$f = \prs{f * 1} * \mu$.

\begin{enumerate}[label = (\alph*)]
\item
מתקיים
\[\text{,} \sum_{d \mid n} \mu\prs{n/d} \nu\prs{d} = \prs{\nu * \mu}\prs{n}\]
ונשים לב כי
\[\text{.} \nu\prs{n} = \sum_{d \mid n} 1 = \prs{1 * 1}(n)\]
אז
\begin{align*}
\text{,} \prs{\nu * \mu}\prs{n} = \prs{1 * 1 * \mu}\prs{n} = 1\prs{n} = 1
\end{align*}
כנדרש.

\item
מתקיים
\begin{align*}
\text{,} \sum_{d \mid n} \mu\prs{n/d} \sigma\prs{d} = \prs{\sigma * \mu}\prs{n}
\end{align*}
ונשים לב כי
\[\text{.} \sigma\prs{n} = \sum_{d \mid n} d = \prs{\id_{\mbb{N}_+} * 1}\prs{n}\]
אז
\begin{align*}
\text{,} \prs{\sigma * \mu}\prs{n} = \prs{\id_{\mbb{N}_+} * 1 * \mu}\prs{n} = \id_{\mbb{N}_+}\prs{n} = n 
\end{align*}
כנדרש.
\end{enumerate}
\end{solution}

\begin{exercisechap}[פרק 2, תרגיל 16]
הראו כי
$\nu\prs{n}$
אי־זוגי אם ורק אם
$n$
ריבוע.
\end{exercisechap}

\begin{solution}
נכתוב
$n = \prod_{i \in \brs{k}} p_i^{r_i}$.
ראינו כי אז
\[\text{.}\nu\prs{n} = \prod_{i \in \brs{k}} \prs{r_i + 1}\]
מספר זה אי־זוגי אם ורק אם כל ה־%
$r_i$
זוגיים, מה שמתקיים אם ורק אם
$n$
ריבוע.
\end{solution}

\begin{exercisechap}[פרק 2, תרגיל 16]
הראו כי
$\sigma\prs{n}$
אי זוגי אם ורק אם
$n$
ריבוע או ריבוע כפול
$2$.
\end{exercisechap}

\begin{solution}

\end{solution}

\begin{exercisechap}[פרק 2, תרגיל 18]
הראו כי
\[\text{.} \forall m,n \in \mbb{N}_+ : \phi\prs{n} \phi\prs{m} = \phi\prs{\gcd\prs{n,m}} \phi\prs{\lcm\prs{n,m}}\]
\end{exercisechap}

\begin{solution}
נזכיר כי עבור
$x = p_1^{a_1} \cdot \ldots p_{\ell}^{a_{\ell}}$
מתקיים באופן כללי
\[\text{.} \phi\prs{x} = x \prod_{k \in \brs{\ell}} \prs{1 - \frac{1}{p_k}}\]

יהיו
\begin{align*}
n &= \prs{\prod_{i \in \brs{k}} p_i^{\alpha_i}} \prs{\prod_{i \in \brs{\ell}} q_i^{r_i}} \\
m &= \prs{\prod_{i \in \brs{k}} p_i^{\beta_i}} \prs{\prod_{i \in \brs{\tilde{\ell}}} \tilde{q}_i^{s_i}}
\end{align*}
הפירוקים של
$n,m$
לראשוניים, כאשר
$p_1, \ldots, p_k$
הראשוניים שמחלקים גם את
$n$
וגם את
$m$.

אז
\begin{align*}
\phi\prs{n} &= n \prs{\prod_{i \in \brs{k}} \prs{1 - \frac{1}{p_i}}} \prs{\prod_{i \in \brs{\ell}} \prs{1 - \frac{1}{q_i}}}
\\
\phi\prs{m} &= m  \prs{\prod_{i \in \brs{k}} \prs{1 - \frac{1}{p_i}}} \prs{\prod_{i \in \brs{\tilde{\ell}}} \prs{1 - \frac{1}{\tilde{q}_i}}}
\\
\phi\prs{\gcd\prs{n,m}} &= \gcd\prs{n,m} \prod_{i \in \brs{k}} \prs{1 - \frac{1}{p_i}}
\\
\phi\prs{\lcm\prs{n,m}} &= \lcm\prs{n,m} \prs{\prod_{i \in \brs{k}} \prs{1 - \frac{1}{p_i}}} \prs{\prod_{i \in \brs{\ell}} \prs{1 - \frac{1}{q_i}}} \prs{\prod_{i \in \brs{\tilde{\ell}}} \prs{1 - \frac{1}{\tilde{q}_i}}}
\end{align*}
וכיוון ש־%
$\gcd\prs{n,m} \lcm\prs{n,m} = nm$,
נקבל כי
\[\text{,}\phi\prs{n} \phi\prs{m} = \phi\prs{\gcd\prs{n,m}} \phi\prs{\lcm\prs{n,m}}\]
כנדרש.
\end{solution}

\chapter{תרגול 4 - עוד פריקות יחידה, וחשבון מודולרי}

\begin{exercisechap}[פרק 2, תרגיל 19]
הראו כי
\[\text{.} \forall m,n \in \mbb{N}_+ : \phi\prs{mn} \phi\prs{\gcd\prs{m,n}} = \gcd\prs{m,n} \phi\prs{m} \phi\prs{n}\]
\end{exercisechap}

\begin{solution}
כיוון שראשוני מחלק את
$mn$
אם ורק אם הוא מחלק את
$\lcm\prs{m,n}$,
נקבל כי
\begin{align*}
\frac{\phi\prs{mn}}{mn} &= \prod_{\substack{p \mid mn \\ \text{\textenglish{$p$ prime}}}} \prs{1 - \frac{1}{p}}
\\&=
\prod_{\substack{p \mid \lcm\prs{m,n} \\ \text{\textenglish{$p$ prime}}}} \prs{1 - \frac{1}{p}}
\\&=
\frac{\phi\prs{\lcm\prs{m,n}}}{\lcm\prs{m,n}} \text{.}
\end{align*}
נקבל כי
\begin{align*}
\phi\prs{mn} &= \frac{mn}{\lcm\prs{m,n}} \cdot \phi\prs{\lcm\prs{m,n}}
\\ \text{.}\hphantom{\phi\prs{mn}} &= \gcd\prs{m,n} \phi\prs{\lcm\prs{m,n}}
\end{align*}
לכן
\begin{align*}
\phi\prs{mn} \phi\prs{\gcd\prs{m,n}} &= \gcd\prs{m,n} \phi\prs{\lcm\prs{m,n}} \phi\prs{\gcd\prs{m,n}}
\\&= \gcd\prs{m,n} \phi\prs{m} \phi\prs{n}
\end{align*}
כאשר בשוויון השני השתמשנו בתרגיל הקודם.
\end{solution}

\begin{exercisechap}[פרק 2, תרגיל 20]
הראו כי
\[\text{.} \prod_{d \mid n} d = n^{\nu\prs{n} / 2}\]

היעזרו בעובדה הבאה:
$\nu\prs{n}$
אי־זוגי אם ורק אם
$n$
ריבוע.
\end{exercisechap}

\begin{solution}
יהי
$n = \prod_{i \in \brs{k}} p_i^{r_i}$
פירוק של
$n$
לראשוניים.

נניח ראשית כי
$n = m^2$
עבור
$m \in \mbb{N}_+$.
מתקיים
\[\text{.} n^{\nu\prs{n} / 2} = \prs{m^2}^{\nu\prs{n} / 2} = m^{\nu\prs{n}}\]
אז
\[\text{,} \ord_{p_i} \prs{ n^{\nu\prs{n}/2} } = \nu\prs{n} \cdot \ord_{p_i}\prs{m}\]
ולכן די להראות כי
\[\text{.} \ord_{p_i} \prs{\prod_{d \mid n}} = \nu\prs{n} \cdot \ord_{p_i}\prs{m}\]
כיוון ש־%
$m^2 = n$,
מתקיים
$\ord_{p_i}\prs{n} = 2 \ord_{p_i}\prs{m}$,
לכן
$\ord_{p_i}\prs{m} = \frac{\ord_{p_i}{n}}{2}$.
לכן די להוכיח כי
\[\text{.} \ord_{p_i} \prs{\prod_{d \mid n} d} = \frac{\nu\prs{n} \cdot \ord_{p_i}\prs{n}}{2}\]

אם
$n$
אינו ריבוע,
$\nu\prs{n}$
זוגי, ואז
$\nu\prs{n} / 2$
שלם. נקבל כי במקרה זה
\[\text{,}\ord_{p_i}\prs{n^{\nu\prs{n} / 2}} = \frac{\nu\prs{n} \cdot \ord_{p_i}\prs{n}}{2}\]
ולכן גם במקרה זה די להוכיח כי
\[\text{.} \ord_{p_i} \prs{\prod_{d \mid n} d} = \frac{\nu\prs{n} \cdot \ord_{p_i}\prs{n}}{2}\]

נקבע
$i \in \brs{k}$.
מתקיים
\begin{align*}
\text{.} \ord_{p_i} \prs{\prod_{d \mid n} d} = \sum_{d \mid n} \ord_{p_i}\prs{d}
\end{align*}
לכל
$r \in \set{0, \ldots, r_i}$
נסמן
\[\text{,} A_r = \set{d}{\substack{d \mid n \\ \ord_{p_i}\prs{d} = r}}\]
ואז
\begin{align*}
\text{.} \set{d}{d \mid n} &= \bigcup_{r = 0}^{r_i} A_r 
\end{align*}
כל הקבוצות
$A_r$
מאותו גודל, כי עבור בחירת החזקה עבור
$p_i$
יש אותו מספר דרכים לבחור את שאר החזקות.
מתקיים גם
$\abs{\bigcup_{r = 0}^{r_i} A_r} = \nu\prs{n}$,
ולכן
\[\abs{A_r} = \frac{\nu\prs{n}}{r_i + 1}\]
לכל
$r \in \set{0, \ldots, r_i}$.
נקבל
\begin{align*}
\sum_{d \mid n} \ord_{p_i}\prs{d} &=
\sum_{r \in \set{0, \ldots, r_i}} \sum_{d \in A_r} r
\\&=
\sum_{r \in \set{0, \ldots, r_i}} \abs{A_r} r
\\&=
\frac{\nu\prs{n}}{r_i + 1} \cdot \sum_{r \in \set{0, \ldots, r_i}} r
\\&= \frac{\nu\prs{n}}{r_i + 1} \cdot \frac{r_i \prs{r_i+1}}{2}
\\&= \frac{\nu\prs{n} r_i}{2}
\\&= \frac{\nu\prs{n} \ord_{p_i}\prs{n}}{2}
\end{align*}
כנדרש.

\end{solution}

\begin{exercisechap}[פרק 3, תרגיל 1]
הראו שיש אינסוף ראשוניים
$p$
עבורם
$p \equiv 5 \mod{6}$.
\end{exercisechap}

\begin{solution}
נניח בדרך השלילה שיש מספר סופי של ראשוניים
$p_1, \ldots, p_\ell$
עבורם
$p_i \equiv 5 \mod{6}$.

אם
$\ell$
אי־זוגי, נגדיר
$m = \prod_{i \in \brs{\ell}} p_i + 6$
ואז
\[m \equiv \prs{-1}^\ell \equiv -1 \mod{6}\]
וזה מספר שזר לכל
$p_i$.
אבל,
$ab \equiv -1 \mod{6}$
גורר שלפחות אחד מבין
$a,b$
שווה
$-1 \mod{6}$.
מפריקות יחידה, נקבל כי
$m$
חייב להתחלק בראשוני ששווה
$-1 \mod{6}$,
בסתירה לכך שהוא לא מתחלק באף
$p_i$.

אם
$\ell$
זוגי, נגדיר במקום זאת
$m = 5 \prod_{i \in \brs{\ell}} p_i + 6$
ונחזור למקרה הקודם.
\end{solution}

\begin{exercisechap}[פרק 3, תרגיל 6]
יהי
$n > 0$.
קבוצת מספרים
$\set{a_1, \ldots, a_{\phi\prs{n}}}$
נקראת
\textbf{מערכת שאריות מצומצמת מודולו $n$}
אם
$\gcd\prs{a_i, n} = 1$
לכל
$i \in \brs{\phi\prs{n}}$
וגם
$a_i \not\equiv a_j \mod{n}$
כאשר
$i \neq j$.

תהי
$R \coloneqq \set{a_1, \ldots, a_{\phi\prs{n}}}$
מערכת שאריות מצומצמת מודולו
$n$
ויהי
$a \in \mbb{Z}$
עבורו
$\gcd\prs{a,n} = 1$.
הראו כי
$aR \coloneqq \set{a a_1, \ldots, a a_{\phi\prs{n}}}$
מערכת שאריות מצומצמת מודולו
$n$.
\end{exercisechap}

\begin{solution}
ראשית, נשים לב כי לכל
$i \in \brs{\phi\prs{n}}$
מתקיים
$\gcd\prs{a a_i, n} = 1$
כי
$a, a_i$
שניהם זרים ל־%
$n$.

נסמן ב־%
$G \coloneqq \prs{\mbb{Z} / n \mbb{Z}}^\times$
את קבוצת השאריות מודולו
$n$
שזרות ל־%
$n$,
ונשים לב\textbackslash%
ניזכר שקבוצה זאת היא חבורה ביחס לכפל.
אכן, עבור
$x \in G$
כיוון שמתקיים
$\gcd\prs{x,n} = 1$
קיימים
$\alpha, \beta \in \mbb{Z}$
עבורם
$\alpha x + \beta n = 1$,
ואז
$\alpha x \equiv 1 \mod{n}$,
כלומר
$\alpha \equiv x^{-1} \mod{n}$.

בפרט, קיים איבר הופכי ל־%
$a$
מודולו
$n$,
ואז
\begin{align*}
\bar{a}^{-1} \overline{a a_1}, \ldots, \bar{a}^{-1} \overline{a a_{\phi\prs{n}}}
&=
\bar{a}^{-1} \bar{a} \bar{a}_1, \ldots, \bar{a}^{-1} \bar{a} \bar{a}_{\phi\prs{n}}
\\ \text{.}\hphantom{\bar{a}^{-1} \overline{a a_1}, \ldots, \bar{a}^{-1} \overline{a a_{\phi\prs{n}}}} &=
\bar{a}_1, \ldots, \bar{a}_{\phi\prs{n}}
\end{align*}
לכן ההעתקה
\begin{align*}
G &\to G \\
x &\mapsto \bar{a}x
\end{align*}
הינה הפיכה, ולכן פרמוטציה. לכן האיברים
$\bar{a} \bar{a}_1, \ldots, \bar{a} \bar{a}_{\phi\prs{n}}$
כולם שונים, כנדרש.
\end{solution}

\chapter{תרגול 5 - עוד חשבון מודולרי}

\begin{exercisechap}[פרק 3, תרגיל 7]
היעזרו בתרגיל הקודם כדי להוכיח את משפט אוילר,
$a^{\phi\prs{n}} \equiv 1 \mod{n}$
כאשר
$\prs{a,n} = 1$.
\end{exercisechap}

\begin{solution}
יהיו
$a \in \mbb{Z}, n \in \mbb{N}_+$
עבורם
$\gcd\prs{a,n} = 1$.
תהי
$\bar{a}$
השארית של
$a$
מודולו
$n$.
ראינו כי
$\prs{\mbb{Z} / n\mbb{Z}}^\times$
חבורה כפלית מסדר
$\phi\prs{n}$,
והחזקה של איבר בסדר של החבורה תמיד שווה ליחידה, לכן
$a^{\phi\prs{n}} \equiv 1 \mod{n}$,
כנדרש.
\end{solution}

\begin{exercisechap}
עבור משולש
$T$
נסמן את אורכי הצלעות בתור
$\ell\prs{T}$.
נגיד כי
$T$
\emph{כמעט שווה צלעות}
אם
$d\prs{T} = \set{a,a,a \pm 1}$
עבור
$n \in \mbb{N}$
כלשהו.

הראו כי אם
$T$
מקיים
$d\prs{T} = \set{a,a, a \pm 1}$
עבור
$a \in \mbb{N}$,
וגם את זה שהשטח של
$T$
שלם,
אז
$a$
אי־זוגי.
\end{exercisechap}

\begin{solution}
נסמן
$b = a \pm 1$
את אורך הצלע השלישית, ונמקם את הקודקוד שמול הצלע הזאת על הראשית, ואת האנך לצלע על ציר ה־%
$x$.

אז אורך האנך הוא
$h = \cos\prs{\alpha} \cdot a$
כאשר
$\alpha$
הזווית מעל ציר ה־%
$x$
מקיימת
$\alpha = \arcsin\prs{\frac{b}{2a}}$.
נקבל כי
\begin{align*}
h &= \cos\prs{\arcsin\prs{\frac{b}{2a}}} \cdot a
\\&= \sqrt{1 - \sin\prs{\arcsin\prs{\alpha}}^2} \cdot a
\\\text{.}\hphantom{h}&= \sqrt{a^2 - \prs{\frac{b}{2}}^2}
\end{align*}
השטח של
$T$
שווה
\[A = \frac{h \cdot b}{2} = \sqrt{\frac{a^2 b^2}{4} - \prs{\frac{b^2}{4}}^2}\]
ולכן
\[\text{.} A^2 = \frac{a^2 b^2}{4} - \prs{\frac{b^2}{4}}^2\]
נכפול ב־%
$16$
ונקבל
\[\text{.} 16 A^2 = 4 a^2 b^2 - b^4\]
מוד
$4$
נקבל
\[\text{,} 0 \equiv -b^4 \mod{4}\]
ולכן
\[b \equiv 0 \mod{4}\]
כלומר,
$b = a \pm 1$
זוגי.
\end{solution}

\begin{exercisechap}[פרק 3, תרגיל 9]
הראו כי
$\prs{p-1}! \equiv -1 \mod{p}$
לכל
$p \in \mbb{N}_+$
ראשוני.
\end{exercisechap}

\begin{solution}
אם
$p = 2$,
הטענה ברורה. לכן נניח
$p \neq 2$.

הביטוי
$\prs{p-1}!$
הוא מכפלת כל האיברים השונים מאפס מודולו
$p$,
כלומר איברי
$\mbb{Z} / p\mbb{Z}$.

כל איבר במכפלה יצתמצם עם ההופכי שלו, אלא אם הוא ההופכי של עצמו.
האיברים
$a \in \mbb{Z} / p\mbb{Z}$
עבורם
$a = a^{-1}$
הם אלו עבורם
$a^2 = 1$.
אלו שורשי הפולינום
$x^2 - 1$,
וכיוון ש־%
$\mbb{Z} / p\mbb{Z}$
שדה, יש לפולינום הזה בדיוק שני שורשים
$\pm 1$.

נקבל כי
$\prs{p-1}! = -1$.
\end{solution}

\begin{exercisechap}[פרק 3, תרגיל 10]
יהי
$n \in \mbb{N}_+$
שאינו ראשוני. הראו כי
\[\prs{n-1}! \equiv 0 \mod{n}\]
חוץ מכאשר
$n = 4$.
\end{exercisechap}

\begin{solution}
נניח ראשית כי
$n = 4$.
אז
$\prs{n-1}! = 3! = 6 \equiv 2 \mod{4}$.

נניח כעת כי
$n \neq 4$.
ניתן לכתוב
$n = ab$
עבור
$a,b \in \set{2, \ldots, n-1}$.
אם
$a \neq b$
נקבל כי
$a,b$
שניהם מופיעים כגורמים במכפלה
$\prs{n-1}!$,
ולכן
$n = ab \mid \prs{n-1}!$
ונקבל כי
$\prs{n-1}! \equiv 0 \mod{n}$.
אחרת,
$n = p^2$
עבור
$p$
ראשוני שונה מ־%
$2$.
נקבל כי
$n = p^2 \mid p \prs{2p} \mid \prs{n-1}!$
ולכן
$\prs{n-1}! \equiv 0 \mod{n}$,
כנדרש.
\end{solution}

\begin{exercisechap}[פרק 3, תרגיל 11]
תהי
$a_1, \ldots, a_{\phi\prs{n}}$
מערכת שאריות מצומצמת מודלו
$n$
ויהי
$N$
מספר הפתרונות למשוואה
$x^2 \equiv 1 \mod{n}$.
הראו כי
\[\text{.} a_1 \cdot \ldots \cdot a_{\phi\prs{n}} \equiv \prs{-1}^{N/2} \mod{n}\]
\end{exercisechap}

\begin{solution}
ראשית, נשים לב כי
$N$
אכן זוגי כי אם
$a^2 \equiv 1 \mod{n}$
גם
$\prs{-a}^2 \equiv 1 \mod{n}$,
ואם
$a \equiv -a \mod{n}$
אז
$2a \equiv 0 \mod{n}$
כלומר
$a$
לא הפיך ב־%
$\mbb{Z} / n\mbb{Z}$,
בסתירה.

כעת, במכפלה
\[\overline{a_1 \cdot \ldots \cdot a_{\phi\prs{n}}}\]
מופיעים איברים וההופכיים שלהם, כאשר ב־%
$N$
מהאיברים המספר שווה להופכי של עצמו.
נניח בלי הגבלת הכלליות שאיברים אלו הם
$a_1, \ldots, a_N$
ונקבל כי
\[\text{,} a_1 \cdot \ldots \cdot a_{\phi\prs{n}} \equiv a_1 \cdot \ldots \cdot a_N \mod{n}\]
כאשר
$a_i^2 = a_i$
לכל
$i \in \brs{N}$.

אם
$x^2 = 1$,
גם
$\prs{-x}^2 = 1$,
ולכן בביטוי
$a_1 \cdot \ldots \cdot a_N$
מופיעות
$N/2$
כפולות של איבר והנגדי שלו.
עבור
$x$
כזה מתקיים
$x \prs{-x} = -x^2 = -1$.
לכן
\[\text{,} a_1 \cdot \ldots \cdot a_N \equiv \prs{-1}^{N/2} \mod{n}\]
כנדרש.
\end{solution}

\begin{exercisechap}[פרק 3, תרגיל 15]
יהי
$p \in \mbb{N}_+$
ראשוני.
הראו כי המונה של
$\sum_{k = 1}^{p-1} \frac{1}{k}$
מתחלק ב־%
$p$.
\end{exercisechap}

\begin{solution}
ניקח מכנה משותף
$\prs{p-1}!$.
מספר זה זר ל־%
$p$
כי
$\prs{p-1}! \equiv -1 \mod{p}$
ממפשט ווילסון.

לכן, המונה בשבר זה מתחלק ב־%
$p$
אם ורק אם המונה בשבר המצומצם מתחלק ב־%
$p$.
המונה יהיה
$\sum_{k = 1}^{p-1} \frac{\prs{p-1}!}{k}$.
\end{solution}

\begin{exercisechap}[פרק 3, תרגיל 17]
יהי
$f\prs{x} \in \mbb{Z}\brs{x}$
ויהי
\[\text{.} n = p_1^{a_1} \cdot \ldots \cdot p_k^{a_k}\]
הראו כי למשוואה
$f\prs{x} \equiv 0 \mod{n}$
יש פתרון
אם ורק אם למשוואה
$f\prs{x} \equiv 0 \mod{p_i^{a_i}}$
יש פתרון
לכל
$i \in \brs{k}$.
\end{exercisechap}

\begin{solution}
נניח כי יש ל־%
$f\prs{x} \equiv 0 \mod{n}$
פתרון
$f\prs{s} \equiv 0 \mod{n}$.
ניקח את המשוואה מוד
$p_i^{a_i}$
לכל
$i \in \brs{k}$
ונקבל
$f\prs{s} \equiv 0 \mod{p_i^{a_i}}$.

נניח כעת כי ל־%
$f\prs{x} \equiv 0 \mod{p_i^{a_i}}$
יש פתרון לכל
$i \in \brs{k}$.
ממשפט השאריות הסיני, כלומר שקיימים
$s_i \in \mbb{Z}$
עבורם
$f\prs{s_i} \equiv 0 \mod{p_i^{a_i}}$.
ממשפט השאריות הסיני, כיוון ש־%
$p_i^{a_i}, p_j^{a_j}$
זרים עבור
$i \neq j$,
קיים
$s \in \mbb{Z}$
עבורו
$s \equiv s_i \mod{p_i^{a_i}}$
לכל
$i \in \brs{k}$.
אז
\[\text{,} f\prs{s} \equiv f\prs{s_i} \mod{p_i^{a_i}} \equiv 0 \mod{p_i^{a_i}}\]
לכל
$i \in \brs{k}$.
\end{solution}

\begin{exercisechap}
יהי
$f\prs{x} \in \mbb{Z}\brs{x}$
ויהי
\[\text{.} n = p_1^{a_1} \cdot \ldots \cdot p_k^{a_k}\]
יהי
$N$
מספר הפתרונות של
$f\prs{x} \equiv 0 \mod{n}$
ויהיו
$N_i$
מספר הפתרונות של
$f\prs{x} \equiv 0 \mod{p_i^{a_i}}$.
הראו כי
\[\text{.} N = \prod_{i \in \brs{k}} N_i\]
\end{exercisechap}

\begin{solution}
בתרגיל הקודם בנינו התאמה חד־חד ערכית ועל בין פתרונות
$f\prs{s} \equiv 0 \mod{n}$
לבין
$\prs{s_1, \ldots, s_k}$
כך ש־%
$f\prs{s_i} \equiv 0 \mod{p_i^{a_i}}$.
מספר הדרכים לבחור פתרונות
$\prs{s_1, \ldots, s_k}$
כאלה הוא
$N_1 \cdot \ldots \cdot N_k$,
ולכן
$N = N_1 \cdot \ldots \cdot N_k$.
\end{solution}

\chapter{תרגול 6 - הדדיות ריבועית}

\section*{תזכורת}

נזכיר טענה מההרצאה.

\begin{proposition}
יהי
$m \in \mbb{N}_+$
עם פירוק לראשוניים
\[\text{.} m = 2^e \cdot \prod_{i \in \brs{\ell}} p_i^{e_i}\]
איבר
$a \in \mbb{Z} / m\mbb{Z}$,
הוא ריבוע אם ורק אם מתקיימים התנאים הבאים
\begin{enumerate}
\item $a \equiv 1 \mod{8}$
אם
$e \geq 3$,
או
$a \equiv 1 \mod{4}$
אם
$e = 2$.
\item לכל
$i \in \brs{\ell}$
מתקיים
$a^{\frac{p_i - 1}{2}} \equiv 1 \mod{p_i}$.
\end{enumerate}
\end{proposition}

נזכיר גם את ההגדרה הבאה, ומספר תכונות לגביה.

\begin{definition}[סימן לז'נדר]
עבור
$p \in \mbb{Z}$
ראשוני, ועבור
$a \in \mbb{N}_+$,
\textbf{סימן לז'נדר}
$\prs{\frac{a}{p}}$
שווה:
\begin{itemize}
\item[-] $0$ אם $p \mid a$.
\item[-] $1$ אם $p \nmid a$ וגם $a$ ריבוע מוד $p$.
\item[-] $1$ אם $p \nmid a$ וגם $a$ אינו ריבוע מוד $p$.
\end{itemize}
\end{definition}

\begin{proposition}
יהיו
$a,b,p,q \in \mbb{Z}$
עבור
$p,q$
ראשוניים חיוביים.
מתקיים:
\begin{enumerate}
\item $a^{\frac{p-1}{2}} \equiv \prs{\frac{a}{p}} \mod{p}$.
\item $\prs{\frac{ab}{p}} = \prs{\frac{a}{p}} \prs{\frac{b}{p}}$.
\item אם
$a \equiv b \mod{p}$
אז
$\prs{\frac{a}{p}} = \prs{\frac{b}{p}}$.
\end{enumerate}
\end{proposition}

\begin{theorem}
יהיו
$p,q$
ראשוניים אי־זוגיים חיוביים שונים. מתקיים
\begin{align*}
\prs{\frac{2}{p}} &= \prs{-1}^{\frac{p^2 - 1}{8}} \\
\text{.} \prs{\frac{p}{q}} \prs{\frac{q}{p}} &= \prs{-1}^{\frac{p-1}{2} \cdot \frac{q-1}{2}}
\end{align*}
\end{theorem}

\begin{corollary}
אם
$q \equiv 1 \mod{4}$
או
$p \equiv 1 \mod{4}$
אז
\[\text{.}\prs{\frac{p}{q}} = \prs{\frac{q}{p}}\]
אחרת,
\[\text{.} \prs{\frac{p}{q}} = - \prs{\frac{q}{p}}\]
\end{corollary}

\section*{תרגילים}

\begin{exercisechap}[פרק 5, תרגיל 1]
חשבו את הביטויים הבאים בעזרת הטענה והמשפט.
\begin{enumerate}
\item $\prs{\frac{5}{7}}$
\item $\prs{\frac{3}{11}}$
\item $\prs{\frac{6}{13}}$
\end{enumerate}
\end{exercisechap}

\begin{solution}
\begin{enumerate}
\item
\begin{align*}
\prs{\frac{5}{7}} &= \prs{\frac{7}{5}}
\\&= \prs{\frac{2}{5}}
\\&= \prs{-1}^{\frac{5^2 - 1}{8}}
\\&= \prs{-1}^3
\\&= -1
\end{align*}
\item
\begin{align*}
\prs{\frac{3}{11}} &= - \prs{\frac{11}{3}}
\\&= - \prs{\frac{2}{3}}
\\&= - \prs{-1}^{\frac{3^2 - 1}{8}}
\\&= 1
\end{align*}
\item
\begin{align*}
\prs{\frac{6}{13}} &= \prs{\frac{2}{13}} \prs{\frac{3}{13}}
\\&= \prs{-1}^{\frac{13^2 - 1}{8}} \prs{\frac{13}{3}}
\\&=  \prs{\frac{13}{3}}
\\&= \prs{\frac{1}{3}}
\\&= 1
\end{align*}
\end{enumerate}
\end{solution}

\begin{exercisechap}[פרק 5, תרגיל 2]
הראו שמספר הפתרונות של המשוואה
$x^2 \equiv a \mod{p}$
הוא
$1 + \prs{\frac{a}{p}}$.
\end{exercisechap}

\begin{solution}
נפריד למקרים.

\begin{enumerate}
\item אם
$p \mid a$,
הפתרון היחיד הוא
$x = 0$,
וגם
$\prs{\frac{a}{p}} = 0$,
לכן מספר הפתרונות הוא
$1 + \prs{\frac{a}{p}} = 1$.

\item אם
$p \nmid a$
ו־%
$a$
ריבוע מוד
$p$,
יש שני פתרונות למשוואה, וגם
$\prs{\frac{a}{p}} = 1$.

\item אם
$p \nmid a$
ו־%
$a$
אינו ריבוע מוד
$p$,
אין פתרונות למשוואה, וגם
$\prs{\frac{a}{p}} = -1$.
\end{enumerate}
\end{solution}

\begin{exercisechap}[פרק 5, תרגיל 3]
יהיו
$a,b,c \in \mbb{Z}$
כך ש־%
$p \nmid a$.
הוכיחו כי מספר הפתרונות של המשוואה
\[ax^2 + bx + c \equiv 0 \mod{p}\]
הוא
$1 + \prs{\frac{b^2 - ac}{p}}$.
\end{exercisechap}

\begin{solution}
לפי נוסחאת השורשים, פתרונות המשוואה הם איברי
\[\text{.} \set{-b \pm \sqrt{b^2 - 4ac}}{2a}\]
גודל הקבוצה הזאת הוא מספר השורשים של
$b^2 - 4ac$,
וזה שווה לפי התרגיל הקודם
$1 + \prs{\frac{b^2 - ac}{p}}$.
\end{solution}

\begin{exercisechap}[פרק 5, תרגיל 4]
הוכיחו כי
\begin{align*}
\text{.} \sum_{a = 1}^{p-1} \prs{\frac{a}{p}} = 0
\end{align*}
\end{exercisechap}

\begin{proof}
ההעתקה
\begin{align*}
\phi \colon \prs{\mbb{Z}/p\mbb{Z}}^\times \to \prs{\mbb{Z}/p\mbb{Z}}^\times
x &\mapsto x^2
\end{align*}
משרה איזומורפיזם של חבורות
\begin{align*}
\text{.} \prs{\mbb{Z} / p\mbb{Z}} / \set{\pm 1} \cong \set{a^2}{a \in \prs{\mbb{Z} / p\mbb{Z}}}
\end{align*}
לכן, אם נסמן
\[S_p \coloneqq \set{a^2}{a \in \prs{\mbb{Z} / p\mbb{Z}}}\]
נקבל כי
$\abs{S_p} = \frac{p-1}{2}$.
אז
\begin{align*}
\sum_{a=1}^{p-1} \prs{\frac{a}{p}} &= \sum_{a \in S_p} \prs{\frac{a}{p}} + \sum_{a \in \brs{p-1} \setminus S_p} \prs{\frac{a}{p}}
\\&=
\sum_{a \in S_p} 1 + \sum_{a \in \brs{p-1} \setminus S_p} \prs{-1}
\\&= \frac{p-1}{2} - \frac{p-1}{2}
\\\text{,}\hphantom{\sum_{a=1}^{p-1} \prs{\frac{a}{p}}}&= 0
\end{align*}
כנדרש.
\end{proof}

\begin{exercisechap}[פרק 5, תרגיל 5]
עבור
$a,b \in \mbb{Z}$
עבורם
$p \nmid a$,
הראו כי
\[\text{.} \sum_{k=0}^{p-1} \prs{\frac{ak + b}{p}} = 0\]
\end{exercisechap}

\begin{solution}
עבור
$k_1, k_2 \in \set{0, \ldots, p-1}$,
מתקיים
\begin{align*}
ak_1 + b \equiv ak_2 + b \mod{p}
\end{align*}
אם ורק אם
\begin{align*}
\text{,} ak_1 \equiv ak_2 \mod{p}
\end{align*}
וכיוון ש־%
$p \mid p$
נקבל שזה מתקיים אם ורק אם
$k_1 = k_2$.
לכן
$b \mod{p}, a + b \mod{p}, 2a + b \mod{p}, \ldots, \prs{p-1}a + b \mod{p}$
הם בדיוק איברי
$\mbb{Z}/p\mbb{Z}$,
כאשר כל אחד מופיע פעם אחת.
כיוון ש־%
$\prs{\frac{0}{p}} = 0$,
נקבל מהתרגיל הקודם את הנדרש.
\end{solution}

\begin{exercisechap}[פרק 5, תרגיל 6]
הראו כי מספר הפתרונות של המשוואה
\[x^2 - y^2 \equiv a \mod{p}\]
הוא
\[\text{.} \sum_{k=0}^{p-1} \prs{1 + \prs{\frac{k^2 + a}{p}}}\]
\end{exercisechap}

\begin{solution}
נכתוב את המשוואה בתור
\[\text{.} x^2 \equiv y^2 + a \mod{p}\]
אז עבור
$y = k$
מספר הפתרונות של המשוואה הוא מספר הפתרונות של המשוואה
\[x^2 \equiv k^2 + a \mod{p}\]
וראינו שזה שווה
$1 + \prs{\frac{k^2 + a}{p}}$.
\end{solution}

\begin{exercisechap}[פרק 5, תרגיל 7]
הראו על ידי חישוב ישיר שמספר הפתרונות של המשוואה
\[x^2 - y^2 \equiv a \mod{p}\]
הוא
$p-1$
אם
$p \nmid a$
או
$2p-1$
אם
$p \mid a$.

\textbf{רמז:}
סמנו
$u = x+y, v = x-y$.
\end{exercisechap}

\begin{solution}
עם הסימונים שהוצעו, נקבל שהמשוואה היא באופן שקול
\[\text{.} uv \equiv a \mod{p}\]

אם
$p \mid a$,
מספר הפתרונות הוא מספר הזוגות
$\prs{u,v}$
עבורם לפחות אחד מבין
$u,v$
שווה
$0$.
זה שווה בדיוק
$2p - 1$
כי כאשר כל אחד שווה אפס, השני יכול להיות כל ערך ב־%
$\mbb{Z} / p \mbb{Z}$,
אבל אז
$\prs{0,0}$
נספר פעמיים.

אם
$p \nmid a$,
לכל בחירה של
$u$
הפיך נקבל כי
$v \equiv \frac{a}{u}$
איבר יחיד הפותר את המשוואה. לכן במקרה זה יש
$p-1$
פתרונות.
\end{solution}

\begin{exercisechap}[פרק 5, תרגיל 8]
היעזרו בשני התרגילים הקודמים כדי להראות כי
\[\text{.} \sum_{k=0}^{p-1} \prs{\frac{k^2 + a}{p}} = \begin{cases}
-1 & p \nmid a \\
p-1 & p \mid a
\end{cases}\]
\end{exercisechap}

\begin{solution}
נניח כי
$p \nmid a$.
אז משני התרגילים נקבל כי
\[\text{,} p + \sum_{k=0}^{p-1} \prs{\frac{k^2 + a}{p}} = \sum_{k=0}^{p-1} \prs{1 + \prs{\frac{k^2 + a}{p}}} = p-1\]
ואז
\[\text{.} \sum_{k=0}^{p-1} \prs{\frac{k^2 + a}{p}} = -1\]

נניח כי
$p \mid a$.
משני התרגילים נקבל כי
\[\text{,} p + \sum_{k=0}^{p-1} \prs{\frac{k^2 + a}{p}} = \sum_{k=0}^{p-1} \prs{1 + \prs{\frac{k^2 + a}{p}}} = 2p-1\]
ואז
\[\text{.} \sum_{k=0}^{p-1} \prs{\frac{k^2 + a}{p}} = p-1\]
\end{solution}

\begin{exercisechap}[פרק 5, תרגיל 10]
יהיו
$r_1, \ldots, r_{\frac{p-1}{2}}$
הריבועים ההפיכים מוד
$p$.
הראו כי
\[
\text{.} \prod_{i \in \brs{\frac{p-1}{2}}} r_i = \begin{cases}
1 & p \equiv 3 \mod{4} \\
-1 & p \equiv 1 \mod{4}
\end{cases}
\]
\end{exercisechap}

\begin{solution}
נרצה לחשב את הביטוי
\[\text{.} P \coloneqq \prod_{k = 1}^{\frac{p-1}{2}} k^2\]
נשים לב כי
$k = -\prs{p-k}$,
לכן
$k^2 = \prs{-1} k \prs{p-k}$
לכל
$k \in \mbb{Z}/p\mbb{Z}$.
אז
\begin{align*}
P &= \prs{-1}^{\frac{p-1}{2}} \cdot \prod_{k = 1}^{\frac{p-1}{2}} k \prs{p-k}
\\&= \prs{-1}^{\frac{p-1}{2}} \cdot \prod_{k=1}^{p-1} k
\\&= \prs{-1}^{\frac{p-1}{2}} \cdot \prs{k-1}!
\\&= \prs{-1}^{\frac{p+1}{2}}
\end{align*}
כאשר בשוויון האחרון השתמשנו במשפט ווילסון.
ביטוי זה שווה
$1$
אם
$p \equiv 3 \mod{4}$
או
$-1$
אם
$p \equiv 1 \mod{4}$.
\end{solution}

\chapter{תרגול 7 - הדדיות }

\begin{exercisechap}[פרק 5, תרגיל 11]
יהי
$p > 3$
ראשוני עבורו
$p \equiv 3 \mod{4}$
וכך שגם
$q \coloneqq 2p + 1$
ראשוני.
הראו כי
$2^{p-1}$
אינו ראשוני.

\textbf{רמז:}
הראו כי
$q \mid 2^{p-1}$.
\end{exercisechap}

\begin{solution}
נכתוב
$p = 4k + 3$
עבור
$k \in \mbb{N}_+$.
אז
\[\text{.} q = 2p+1 = 8k + 7 \equiv -1 \mod{8}\]
לכן
\[\prs{\frac{2}{q}} = \prs{-1}^{\frac{q^2 - 1}{8}} = 1\]
אז
\[2^{\frac{q-1}{2}} \equiv 1 \mod{q}\]
כלומר
\[\text{.} 2^p \equiv 1 \mod{q}\]
נקבל כי
$q \mid 2^p - 1$,
ולכן
$2^p - 1$
אינו ראשוני.
\end{solution}

\begin{exercisechap}[פרק 5, תרגיל 16]
מיצאו בעזרת הדדיות ריבועית את הראשוניים עבורם $7$ הוא ריבוע ב־%
$\mbb{Z} / p \mbb{Z}$.
\end{exercisechap}

\begin{solution}
ראשית, $p = 2$
עונה על הדרישה כי
$7 \equiv 1 \mod{2}$.
נניח בהמשך כי
$p$
אי־זוגי.

מתקיים כי
$7$
ריבוע מוד
$p$
אם ורק אם
$p = 7$
או
$p \neq 7$
וגם
$\prs{\frac{7}{p}} = 1$.
נניח כי
$p \neq 7$.
לפי הדדיות ריבועית,
\begin{align*}
\prs{\frac{7}{p}} &= \prs{-1}^{\frac{7-1}{2} \cdot \frac{p-1}{2}} \prs{\frac{p}{7}}
\\
\text{.} \hphantom{\prs{\frac{7}{p}}}
&=
\prs{-1}^{\frac{3 \prs{p-1}}{2}} \prs{\frac{p}{7}}
\end{align*}
אם
$p \equiv 1 \mod{4}$
נקבל כי
$\frac{3 \prs{p-1}}{2}$
זוגי ואז
\begin{align*}
\text{.} \prs{\frac{7}{p}} = \prs{\frac{p}{7}}
\end{align*}
אחרת, נקבל כי
\[\text{.} \prs{\frac{7}{p}} = - \prs{\frac{p}{7}}\]

לכן התשובה היא ראשוניים
$p \in \set{2,7}$
וגם
$p \notin \set{2,7}$
עבורם
\[\prs{\frac{p}{7}} = 1, \quad p \equiv 1 \mod{4}\]
או
\[\text{.}\prs{\frac{p}{7}} = -1, \quad p \equiv 3 \mod{4}\]

נחפש את ערכי
$p$
שמקיימים זאת.
מוד
$7$
מתקיים
\begin{align*}
1^2 &= 1 \\
2^2 &= 4 \\
3^2 &= 2 \\
4^2 &= 2 \\
5^2 &= 4 \\
\text{.} 6^2 &= 1
\end{align*}

נחפש קודם ערכי
$p$
עבורם
$\prs{\frac{p}{7}} = 1, \quad p \equiv 1 \mod{4}$.
נכתוב
$a \equiv p \mod{7} \in \set{1,2,4}$.
מתקיים
$\gcd\prs{4,7} = 1$
ונוכל לכתוב
$1 = 2 \cdot 4 + \prs{-1} \cdot 7$.
אז
$p$
שפותר את המערכת
\begin{align*}
p &\equiv 1 \mod{4} \\
p &\equiv a \mod{7}
\end{align*}
הוא כל ראשוני מהצורה
\begin{enumerate}
\item $1 + 28k$
אם
$a = 1$.
\item $9 + 28k$
אם
$a = 2$.
\item $25 + 28k$
אם
$a = 4$.
\end{enumerate}

באופן דומה, עבור
$p \equiv 3 \mod{4}$
נצטרך למצוא פתרונות ראשוניים למערכת
\begin{align*}
p &\equiv 3 \mod{4} \\
p &\equiv a \mod{7}
\end{align*}
כאשר
$a \in \set{3,5,6}$.
אלו יהיו כל הראשוניים מהצורה
\begin{enumerate}
\item $3 + 28k$
עבור
$a = 3$.
\item $19 + 28k$
עבור
$a = 5$.
\item $27 + 28k$
עבור
$a = 6$.
\end{enumerate}
\end{solution}

\begin{definition}[סימן יקובי]
עבור
$b$
חיובי אי־זוגי עם פירוק לראשוניים
$b = p_1^{a_1} \cdot \ldots p_k^{a_k}$,
נגדיר את
\emph{סימן יקובי}
$\prs{\frac{a}{b}}$
בתור
\begin{align*}
\text{.} \prs{\frac{a}{b}} \coloneqq \prs{\frac{a}{p_i}}^{a_i} \cdot \ldots \cdot \prs{\frac{a}{p_k}}^{a_k}
\end{align*}
\end{definition}

\begin{proposition}
\begin{enumerate}
\item $\prs{\frac{a_1}{b}} = \prs{\frac{a_2}{b}}$
אם
$a_1 \equiv a_2 \mod{b}$.
\item $\prs{\frac{a_1 a_2}{b}} = \prs{\frac{a_1}{b}} \prs{\frac{a_2}{b}}$.
\item $\prs{\frac{a}{b_1 b_2}} = \prs{\frac{a}{b_1}} \prs{\frac{a}{b_2}}$
\end{enumerate}
\end{proposition}

\begin{proposition}
\begin{enumerate}
\item $\prs{\frac{-1}{b}} = \prs{-1}^{\frac{b-1}{2}}$.
\item $\prs{\frac{2}{b}} = \prs{-1}^{\frac{b^2 - 1}{8}}$.
\item אם
$a,b$
שניהם אי זוגיים,
\begin{align*}
\text{.} \prs{\frac{a}{b}} \prs{\frac{b}{a}} = \prs{-1}^{\frac{a - 1}{2} \cdot \frac{b-1}{2}}
\end{align*}
\end{enumerate}
\end{proposition}

\begin{exercisechap}[פרק 5, תרגיל 18]
יהי
$D$
שלם חיובי חסר־ריבועים. הראו שיש
$b$
שלם חיובי זר ל־%
$D$
עבורו
$\prs{\frac{b}{D}} = -1$.
\end{exercisechap}

\begin{solution}
נכתוב
$D = \prod_{i \in \brs{k}} p_i$
עבור
$p_i$
ראשוניים זרים בזוגות.
אז
\begin{align*}
\text{.} \frac{b}{D} = \prod_{i \in \brs{k}} \prs{\frac{b}{p_i}}
\end{align*}
יהיו
$a_i \in \mbb{Z} / p_i \mbb{Z}$
שונים מאפס כך ש־%
$a_1$
ריבוע ב־%
$\mbb{Z} / p_1 \mbb{Z}$
ו־%
$a_i$
אינו ריבוע ב־%
$\mbb{Z} / p_i \mbb{Z}$
לכל
$i \geq 2$.
ממשפט השאריות הסיני קיים
$b \in \mbb{Z}$
עבורו
$b \equiv a_i \mod{p_i}$
לכל
$i \in \brs{k}$.
אז
\[\text{.} \prs{\frac{b}{p_i}} \equiv \prs{\frac{a_i}{p_i}} = \prs{-1}^{\delta_{i,1}}\]
לכן
\begin{align*}
\text{,} \frac{b}{D} = \prod_{i \in \brs{k}} \prs{\frac{b}{p_i}} = -1
\end{align*}
כנדרש.
\end{solution}

\begin{exercisechap}[פרק 5, תרגיל 19]
תהי
$\prs{a_1, \ldots, a_{\phi\prs{D}}}$
מערכת שאריות מצומצמת מוד
$D$.
ת שאריות מצומצמת מוד
$D$.

\begin{enumerate}
\item הראו כי
\begin{align*}
\text{.} \sum_{i \in \brs{\phi\prs{D}}} \prs{\frac{a_i}{D}} = 0
\end{align*}

היזכרו כי לכל
$a$
הפיך מוד
$D$, גם
$\prs{a a_1, \ldots, a a_{\phi\prs{D}}}$
מערכת שאריות מצומצמת מוד
$D$.

\item
הסיקו כי בדיוק חצי מאיברי
$\prs{\mbb{Z} / D\mbb{Z}}^\times$
מקיימים
 $\prs{\frac{a}{D}} = 1$.
\end{enumerate}
\end{exercisechap}

\begin{solution}
\begin{enumerate}
\item מתקיים
\begin{align*}
\text{.} \sum_{i \in \brs{\phi\prs{D}}} \prs{\frac{a_i}{D}} \in \set{1, 0, -1}
\end{align*}
ניקח
$b \in \mbb{Z}$
עבורו
$\prs{\frac{b}{D}} = -1$,
שקיים מהתרגיל הקודם.
אז
\begin{align*}
\sum_{i \in \brs{\phi\prs{D}}} \prs{\frac{a_i}{D}} &=
\sum_{i \in \brs{\phi\prs{D}}} \prs{\frac{b a_i}{D}}
\\&=
\sum_{i \in \brs{\phi\prs{D}}} \prs{\frac{b}{D}} \prs{\frac{a_i}{D}}
\\&=
- \sum_{i \in \brs{\phi\prs{D}}} \prs{\frac{a_i}{D}}
\end{align*}
ולכן
\[\text{,} \sum_{i \in \brs{\phi\prs{D}}} \prs{\frac{a_i}{D}} = 0\]
כנדרש.

\item
כיוון שכל הגורמים
$\prs{\frac{a_i}{D}}$
בסכום הם אחד מבין
$1, -1$,
וכיוון ש־%
$\phi\prs{D} < D$,
נקבל כי הסכום יכול להיות אפס אם ורק אם חצי מהגורמים בו שווים
$1$
וחצי מהם שווים
$-1$.
כיוון שמחלקות השקילות של
$a_1, \ldots, a_{\phi\prs{D}}$
הן איברי
$\mbb{Z} / D\mbb{Z}$,
נקבל את הנדרש.
\end{enumerate}
\end{solution}

\begin{exercisechap}[פרק 5, תרגיל 20]
יהיו
$a_1, \ldots, a_{\phi\prs{D} / 2} \in \mbb{Z} / D\mbb{Z}$
האיברים עבורם
$\prs{\frac{a_i}{D}} = 1$.
יהי
$p \nmid D$
ראשוני שמקיים
$p \equiv 1 \mod{4}$.
הראו כי
$D$
שארית ריבועית מוד
$p$
אם ורק אם קיים
$i \in \brs{\phi\prs{D}}$
עבורו
$p \equiv a_i \mod{D}$.
\end{exercisechap}

\begin{solution}
$D$
שארית ריבועית מוד
$p$
אם ורק אם
$\prs{\frac{D}{p}} = 1$.

נניח כי
$p \equiv a_i \mod{D}$
עבור איזהו
$i \in \brs{\phi\prs{D} / 2}$.
אז
\begin{align*}
\text{.} \prs{\frac{p}{D}} = \prs{\frac{a_i}{D}} = 1
\end{align*}
כיוון ש־%
$p \equiv 1 \mod{4}$,
נקבל כי
$\frac{p-1}{2}$
זוגי ולכן
\begin{align*}
\text{.} \prs{\frac{D}{p}} = \prs{\frac{p}{d}} = 1
\end{align*}

נניח כעת כי
$p \not\equiv a_i \mod{D}$
לאף
$i \in \brs{\phi\prs{D}/2}$.
אז
$\prs{\frac{p}{D}} = -1$
ובאותו אופן נקבל כי
$\prs{\frac{D}{p}} = \prs{\frac{p}{D}} = -1$,
לכן
$D$
לא שארית ריבועית מוד
$p$.
\end{solution}

\begin{comment}

\begin{exercisechap}[פרק 5, תרגיל 21]
השתמשו בשיטה משני התרגילים הקודמים כדי למצוא את כל הראשוניים
$p$
עבורם
$21$
הוא שארית ריבועית מוד
$p$.
\end{exercisechap}

\begin{solution}
ההפיכים מוד
$21$
הם המספרים מ־%
$1$
עד
$20$
שזרים ל־%
$3,7$.
אלו
$1, 2, 4, 5, 8, 10, 11, 13, 16, 17, 19, 20$.

מתקיים
\begin{align*}
\prs{\frac{1}{21}} &= \prs{\frac{1}{3}} \prs{\frac{1}{7}}
\\&= 1
\\
\prs{\frac{2}{21}} &= \prs{-1}^{\frac{21^2 - 1}{8}}
\\&= \prs{-1}^{55}
\\&= -1
\\
\prs{\frac{4}{21}} &= \prs{\frac{2}{21}}^2
\\&=
1
\\
\prs{\frac{5}{21}} &= \prs{\frac{21}{5}}
\\&=
\prs{\frac{1}{5}}
\\&=
1
\\
\prs{\frac{8}{21}} &= \prs{\frac{2}{21}}^3
\\&=
-1
\\
\prs{\frac{10}{21}} &= \prs{\frac{2}{21}} \prs{\frac{5}{21}}
\\&= -1 \cdot 1
\\&= -1
\\
\prs{\frac{11}{21}} &= \prs{\frac{21}{11}}
\\&= \prs{\frac{10}{11}}
\\&= \prs{\frac{2}{11}} \prs{\frac{3}{11}}
\\&= - \prs{-1}^{\frac{11^2 - 1}{8}} \prs{\frac{11}{3}}
\\&= - \prs{-1}^{15} \prs{\frac{2}{3}}
\\&= \prs{-1}^3
\\&= -1
\end{align*}
\begin{align*}
\prs{\frac{13}{21}} &= \prs{\frac{21}{13}}
\\&= \prs{\frac{8}{13}}
\\&= \prs{\frac{2}{13}}^3
\\&= \prs{-1}^{3 \cdot \frac{13^2 - 1}{8}}
\\&= \prs{-1}^{3 \cdot 21}
\\&= -1
\\
\prs{\frac{16}{21}} &= \prs{\frac{2}{21}}^4
\\&= 1
\\
\prs{\frac{17}{21}} &= \prs{\frac{21}{17}}
\\&= \prs{\frac{4}{17}}
\\&= \prs{\frac{2}{17}}^2
\\&= 1
\\
\prs{\frac{19}{21}} &= \prs{\frac{21}{19}}
\\&= \prs{\frac{2}{19}}
\\&= \prs{-1}^{\frac{19^2 - 1}{8}}
\\&= \prs{-1}^{45}
\\&= -1
\\
\prs{\frac{20}{21}} &= \prs{\frac{2}{21}} \prs{\frac{10}{21}}
\\&= \prs{-1}^2
\\&= 1
\end{align*}
ולכן האיברים
$a \in \mbb{Z} / 21 \mbb{Z}$
עבורם
$\prs{\frac{a}{21}} = 1$
הם
$1, 4, 5, 16, 17, 20$.
נשים לב שזה מספר האיברים האלו שאנו אמורים לקבל, כי
$\frac{\phi\prs{21}}{2} = \frac{12}{2} = 6$.

לכן
$21$
שארית ריבועית מוד
$p$
אם ורק אם
$\bar{p} \mod{21} \in \set{1,4,5,16,17,20}$.
\end{solution}

\begin{exercisechap}[פרק 5, תרגיל 23]
יהי
$p \equiv 1 \mod{4}$
ראשוני.

\begin{enumerate}
\item הראו שקיימים
$s,t \in \mbb{Z}$
עבורם
$pt = 1 + s^2$.

\item הסיקו כי
$p$
אינו ראשוני ב־%
$\mbb{Z}\brs{i}$.

זיכרו כי יש ב־%
$\mbb{Z}\brs{i}$
פירוק יחיד לראשוניים.
\end{enumerate}
\end{exercisechap}

\begin{solution}
\begin{enumerate}
\item
מכך ש־%
$p \equiv 1 \mod{4}$
נקבל כי
\[\prs{\frac{-1}{p}} = \prs{-1}^{\frac{p-1}{2}} = 1\]
ולכן
$\prs{-1}$
ריבוע מוד
$p$.
לכן יש
$s \in \mbb{Z}$
עבורו
$s^2 \equiv -1 \mod{p}$,
כלומר
$s^2 + 1 \equiv 0 \mod{p}$,
כלומר קיים
$t \in \mbb{Z}$
עבורו
$pt = s^2 + 1$.

\item 
נקבל כי עבור
$s,t$
שמצאנו מתקיים
$pt = \prs{s+i}\prs{s-i}$.
אם
$p$
היה ראשוני ב־%
$\mbb{Z}\brs{i}$
היינו מקבלים
$p \mid s+i$
או
$p \mid s-i$.
אבל אז
$p \mid 1$
או
$p \mid -1$,
מה שלא יתכן.
\end{enumerate}
\end{solution}
\end{comment}

\chapter{תרגול 8}

\section{מספרים אלגבריים וסכומי גאוס}

\begin{definition}
מספר מרוכב
$\alpha$
הוא אלגברי אם קיים
$p\prs{x} \in \mbb{Q}\brs{x}$
עבורו
$p\prs{\alpha} = 0$.
הוא שלם אלגברי אם קיים
$p \in \mbb{Z}\brs{x}$
עבורו
$p\prs{\alpha} = 0$.
\end{definition}

\begin{proposition}
אוסף המספרים האלגבריים הוא שדה, ואוסף השלמים האלגבריים הוא חוג.
\end{proposition}

\begin{proposition}
לכל
$\alpha, \beta$
שלמים אלגבריים, ולכל
$p$
ראשונים, מתקיים
\begin{align*}
\text{.} \prs{\alpha + \beta}^p \equiv \alpha^p + \beta^p \mod{p}
\end{align*}
\end{proposition}

\begin{definition}
יהי
$\zeta \coloneqq e^{2 \pi i}{8}$
שורש יחידה פרימיטיבי מסדר 8.
עבור בחירה של ראשוני
$p \in \mbb{N}_+$
ועבור
$a \in \brs{p}$,
נגדיר
\begin{align*}
\text{.} g_a \coloneqq \sum_{t \in \brs{p}} \prs{\frac{t}{p}} \zeta^{at}
\end{align*}
\end{definition}

\begin{exercisechap}[פרק 6, תרגיל 8]
יהי
$\omega \coloneqq e^{\frac{2 \pi i}{3}}$
שורש יחידה פרימיטיבי מסדר
$3$.
הראו כי מתקיים
$\prs{2 \omega + 1}^2 = -3$,
וחשבו בעזרת זה את
$\prs{\frac{-3}{p}}$.
\end{exercisechap}

\begin{solution}
ראשית, מתקיים
\begin{align*}
\prs{2 \omega + 1}^2 &= 4 \omega^2 + 4 \omega + 1
\\&= 4 \prs{\bar{\omega} + \omega} + 1
\\&= 8 \Re{\omega} + 1
\end{align*}
וגם
\begin{align*}
\text{,} \Re\prs{\omega} = \cos\prs{\frac{2 \pi}{3}} = -\frac{1}{2}
\end{align*}
לכן
\[\text{.} \prs{2 \omega + 1}^2 = 1 - 4 = -3\]

כעת, נסמן
\[\upsilon \coloneqq 2 \omega + 1\]
בדומה לחישוב מההרצאה עבור
$\prs{\frac{2}{p}}$,
ונקבל כי
$\upsilon^2 = -3$.
לפי טענה מההרצאה,
\begin{align*}
\text{,} \prs{\frac{-3}{p}} \equiv \prs{-3}^{\frac{p-1}{2}} \mod{p}
\end{align*}
ולכן
\begin{align*}
\text{.} \prs{\frac{-3}{p}} &\equiv \upsilon^{p-1} \mod{p}
\end{align*}
לכן
\begin{align*}
\text{.} \prs{\frac{-3}{p}} \cdot \upsilon \equiv \upsilon^p \mod{p}
\end{align*}
מצד שני,
\begin{align*}
\upsilon^p &\equiv \prs{2 \omega + 1}^p
\\& \equiv 2^p \omega^p + 1^p \mod{p}
\\\text{.} \hphantom{\upsilon^p} & \equiv 2 \omega^p + 1 \mod{p}
\end{align*}

אז
\begin{enumerate}
\item אם
$p \equiv 0 \mod{3}$
נקבל כי
$p = 3$
ולכן
$\prs{\frac{-3}{p}} = 0$.

\item אם
$p \equiv 1 \mod{3}$,
נקבל כי
\[\text{.} \upsilon^p = 2 \omega^p + 1 = 2 \omega + 1 = \upsilon\]
לכן במקרה זה
$\prs{\frac{-3}{p}} \equiv 1 \mod{p}$
ולכן
$\prs{\frac{-3}{p}} = 1$.

\item אם
$p \equiv 2 \mod{3}$,
נקבל כי
\begin{align*}
\upsilon^p &\equiv 2 \omega^2 + 1 \mod{p}
\\&\equiv 2 \prs{-1 - \omega} + 1 \mod{p}
\\&\equiv -2 \omega - 1 \mod{p}
\\&\equiv - \upsilon \mod{p}
\end{align*}
ולכן
\begin{align*}
\text{.} \prs{\frac{-3}{p}} = -1
\end{align*}
\end{enumerate}

חישוב יראה לנו שמתקיים ממה שהראנו
$\prs{\frac{-3}{p}} = \prs{\frac{p}{3}}$.
\end{solution}

\begin{proposition}
מתקיים
\begin{align*}
\text{.} g_a = \prs{\frac{a}{p}} g_1
\end{align*}
\end{proposition}

\begin{exercisechap}[פרק 6, תרגיל 10]
ניעזר בטענה על חישוב
$g_a$
ונקבל
\begin{align*}
\sum_{a \in \brs{p-1}} g_a &= \sum_{a \in \brs{p-1}} \prs{\frac{a}{p}} g_1
\\\text{.}\hphantom{\sum_{a \in \brs{p-1}} g_a}&= g_1 \sum_{a \in \brs{p-1}} \prs{\frac{a}{p}}
\end{align*}
כיוון שראינו בתרגול שהסכום באגף ימין מתאפס, נקבל כי הסכום הנדרש שווה אפס.
\end{exercisechap}

\section{סכומי גאוס ויקובי מעל
$\mbb{F}_p$}

\begin{definition}
קרקטר כפלי של
$\mbb{F}_p$
הוא הומומורפיזם
\begin{align*}
\text{.} \chi \colon \mbb{F}_p \to \mbb{C}
\end{align*}
במקרה זה נרחיב את
$\chi$
ל־%
$\mbb{F}_p$.
\end{definition}

\begin{proposition}
אוסף הקרקטרים הכפליים של
$\mbb{F}_p$,
שנסמנו
$\Omega_p$,
הוא חבורה ביחס לכפל.
\end{proposition}

\begin{proposition}
\begin{enumerate}
\item
לכל
$a \in \mbb{F}_p^\times$
שונה מ־%
$1$
מתקיים
\begin{align*}
\text{.} \sum_{\chi \in \Omega_p} \chi\prs{a} = 0
\end{align*}
\item
לכל
$\chi \in \Omega_p$
מתקיים
\begin{align*}
\text{.} \sum_{t \in \mbb{F}_p} \chi\prs{t} = \begin{cases}
0 & \chi \neq 1 \\
p & \chi = 1
\end{cases}
\end{align*}
\end{enumerate}
\end{proposition}

\begin{definition}[סכום גאוס]
עבור
$\chi \in \Omega_p, a \in \mbb{F}_p$,
נגדיר
\begin{align*}
\text{,} g_a\prs{\chi} = \sum_{t \in \mbb{F}_p} \chi\prs{t} \zeta^{at}
\end{align*}
כאשר
$\zeta \coloneqq e^{\frac{2 \pi i}{p}}$.

נסמן
$g = g_1$.
\end{definition}

\begin{proposition}
מתקיים
\begin{align*}
\text{.} g_a =
\begin{cases}
	\chi\prs{a^{-1}} g\prs{\chi} & a \neq 0, \chi \neq 1 \\
	p & a = 0, \chi = 1 \\
	0 & \text{otherwise}
\end{cases}
\end{align*}
\end{proposition}

\begin{definition}[סכום יקובי]
עבור
$\chi, \lambda \in \Omega_p$,
נסמן
\[\text{.} J\prs{\chi, \lambda} = \sum_{a+b = 1} \chi\prs{a} \lambda\prs{b}\]
\end{definition}

\begin{proposition}
מתקיים
\begin{align*}
\text{.} J\prs{\chi, \lambda} =
\begin{cases}
	p & \chi = \lambda = 1 \\
	-\chi\prs{-1} & \lambda = \chi^{-1} \\
	\frac{g\prs{\chi} g\prs{\lambda}}{g\prs{\chi \lambda}} & \chi \lambda \neq 1
\end{cases}
\end{align*}
\end{proposition}

\begin{exercisechap}[פרק 8, תרגיל 3]
יהי
$\chi \in \Omega_p \setminus \set{1}$,
ויהי
$\rho \in \Omega_p$
מסדר
$2$.
הראו כי
\begin{align*}
\text{.} \sum_{t \in \mbb{F}_p} \chi\prs{1 - t^2} = J\prs{\chi, \rho}
\end{align*}

\textbf{רמז:}
היעזרו במשוואה
$N\prs{x^2 = a} = 1 + \rho\prs{a}$
כאשר
$N\prs{E}$
הוא מספר הפתרונות למשוואה
$E$
מעל
$\mbb{F}_p$.
\end{exercisechap}

\begin{solution}
נחשב בעזרת הרמז.
\begin{align*}
J\prs{\chi, \rho} &= \sum_{a+b = 1} \chi\prs{a} \rho\prs{b}
\\&= \sum_{a+b = 1} \chi\prs{a} \prs{N\prs{x^2 = b} - 1}
\\&= \sum_{a + b = 1} \chi\prs{a} N\prs{x^2 = b} - \sum_{a+b = 1} \chi\prs{a}
\end{align*}
כיוון ש־%
$\chi \neq 1$,
נקבל כי
\[\text{.} \sum_{a+b} \chi\prs{a} = \sum_{a \in \mbb{F}_p} \chi\prs{a} = 0\]
אז
\begin{align*}
J\prs{\chi, \rho} &= \sum_{a + b = 1} \chi\prs{a} N\prs{x^2 = b}
\\&= \chi\prs{1} + \sum_{\substack{a + b = 1 \\ \prs{\frac{b}{p}} = 1}} \chi\prs{a} N\prs{x^2 = b} + \cancelto{0}{\sum_{\substack{a + b = 1 \\ \prs{\frac{b}{p}} = -1}} \chi\prs{a} N\prs{x^2 = b}}
\\&= \chi\prs{1} + 2 \sum_{\substack{\prs{\frac{b}{p}} = 1}} \chi\prs{1-b}
\\&= \chi\prs{1} + \sum_{t \in \mbb{F}_p^\times} \chi\prs{1 - t^2}
\\ \text{,} \hphantom{J\prs{\chi, \rho}} &= \sum_{t \in \mbb{F}_p} \chi\prs{1 - t^2}
\end{align*}
כנדרש.
\end{solution}

\begin{comment}

\begin{exercisechap}[פרק 8, תרגיל 5]
יהי
$\chi \in \Omega_p \setminus \set{1}$,
ויהי
$\rho \in \Omega_p$
מסדר
$2$.
יהי
$a \in \mbb{F}_p^\times$.
ידוע כי
\[\text{.} \sum_{t \in \mbb{F}_p} \chi\prs{t\prs{a-t}} = \chi\prs{\frac{a^2}{2^2}} J\prs{\chi, \rho}\]
הראו שאם
$\chi^2 \neq 1$,
מתקיים
\[\text{.} g\prs{\chi}^2 = \chi\prs{2}^{-2} J\prs{\chi, \rho} g\prs{\chi^2}\]

\textbf{רמז:}
כיתבו את
$g\prs{\chi}^2$
מפורשות והיעזרו בנתון.
\end{exercisechap}

\begin{solution}
נחשב.
\begin{align*}
g\prs{\chi}^2 &= \prs{\sum_{t \in \mbb{F}_p} \chi\prs{t} \zeta^t} \prs{\sum_{s \in \mbb{F}_p} \chi\prs{s} \zeta^s}
\\&=
\sum_{s,t \in \mbb{F}_p} \chi\prs{t} \chi\prs{s} \zeta^{t+s}
\\&=
\sum_{k \in \mbb{F}_p} \prs{\sum_{s+t = k} \chi\prs{t} \chi\prs{s}} \zeta^k
\\&=
\sum_{k \in \mbb{F}_p} \prs{\sum_{t \in \mbb{F}_p} \chi\prs{t\prs{k-t}}} \zeta^k
\\&= \sum_{t \in \mbb{F}_p} \chi\prs{-t^2} + \sum_{k \neq 0} \chi\prs{\frac{k^2}{2^2}} J\prs{\chi, \rho} \zeta^k
\\&=
\chi\prs{-1} \sum_{t \in \mbb{F}_p} \chi^2\prs{t} + \chi\prs{2}^{-2} J\prs{\chi, \rho} \sum_{k \neq 0} \chi^2\prs{k} \zeta^k
\\&=
\chi\prs{2}^{-2} J\prs{\chi, \rho} g\prs{\chi^2}
\end{align*}
\end{solution}

\end{comment}

\begin{exercisechap}[פרק 8, תרגיל 6]
הראו כי עבור
$\chi \in \Omega_p$
שמקיים
$\chi^2 \neq 1$,
ועבור
$\rho \in \Omega_p$
מסדר
$2$,
מתקיים
$J\prs{\chi, \chi} = \chi\prs{2}^{-2} J\prs{\chi, \rho}$.
\end{exercisechap}

\begin{solution}
המשפט נותן לנו
$J\prs{\chi, \chi} = \frac{g\prs{\chi}^2}{g\prs{\chi^2}}$,
ומהתרגיל הקודם נקבל כי זה שווה
$\chi\prs{2}^{-2} J\prs{\chi, \rho}$.
לכן
\[\text{.} J\prs{\chi, \chi} = \chi\prs{2}^{-2} J\prs{\chi, \rho}\]
\end{solution}

\printbibliography
\end{document}
